%! TEX root = ../main.tex
\documentclass[main]{subfiles}

\begin{document}

\chapter{Reductions}
\label{chap:reduction}

\section{$E_{0,u}^{(1 + 3u)}$}
We denote by $\mathcal{E}_{0,u}^{(1 + 3u)} \to \mathbb{P}^1$ the elliptic surface with the generic fiber $E_{0,u}^{(1 + 3u)}$.

Table~\ref{tab:E_{0,u}^{(1 + 3u)}} より K3 なので
\begin{equation}
    \dim_{\mathbb{Q}_{l}} H_{\text{\'et}}^{2}(\tilde{S}, \mathbb{Q}_{l}) = 22
\end{equation}
である.
Let $V$ be the subspace of $\NS(\tilde{S})$ generated by the singular fibers and the zero section.
Then $V$ is of rank 19, on which the Frobenius automorphism acts by multiplication by $p$.

\begin{equation}
    \chara (\Phi_{\tilde{S}}^{*} | V) = (x - 5)^{19}
\end{equation}

Note that all the multiplicative fibers are split.

\begin{equation}
    t_{m} := \Tr((\Phi_{\tilde{S},H_{\text{\'et}}^{2} / V}^{*})^{m}) = \# \tilde{S}(\mathbb{F}_{5^{m}}) - 1 - 5^{2m} - 19 \cdot 5^{m}
\end{equation}

\begin{table}[h]
    \centering
    % \caption{Sample Table}
    \begin{tabular}{|c|c|c|c|}
        \hline
        m                                & 1   & 2    & 3     \\
        \hline
        $\# \tilde{S}(\mathbb{F}_{5^m})$ & 120 & 1080 & 18264 \\
        \hline
        $t_m$                            & -1  & -21  & 263   \\
        \hline
    \end{tabular}
    \label{tab:sample}
\end{table}

\begin{equation}
    \chara(\Phi_{\tilde{S},H_{\text{\'et}}^{2} / V}^{*}) = x^{3} + x^{2} + 11 x - 77
\end{equation}

\end{document}
