%! TEX root = ../main.tex
\documentclass[main]{subfiles}

\begin{document}

\chapter{Reductions}
\label{chap:reduction}

Let $A$ be a discrete valuation ring of a number field $K$ with maximal ideal $\mathfrak{m}$, whose residue field $k$ has $q=p^r$ elements with $p$ prime.
Let $S$ be an integral scheme with a morphism $S \to \Spec A$ that is projective and smooth of relative dimension $2$.
Then the projective surface $\overline{S}=S_{\overline{\mathbb{Q}}}$ and $\tilde{S}=S_{\overline{k}}$ are smooth over the algebraically closed field $\overline{\mathbb{Q}}$ and $\overline{k}$, respectively.
We will assume that $\overline{S}$ and $\tilde{S}$ are integrals, i.e., they are irreducible, nonsingular, projective surfaces.

For $l \neq p$ be a prime number, we denote by $H_{\text{\'et}}^{2}(\tilde{S}, \mathbb{Q}_l)$ the $l$-adic \'etale cohomology group of $X$ and by $H_{\text{\'et}}^{2}(\tilde{S}, \mathbb{Q}_l)(1)$ the Tate twist of it.

\begin{thm}{(\cite[Proposition 6.2.]{ref:vanluijk2007})}

    There are natural injective homomorphisms
    \begin{equation}
        \NS (\overline{S}) \otimes_{\mathbb{Z}} \mathbb{Q}_{l} \hookrightarrow \NS (\tilde{S}) \otimes_{\mathbb{Z}} \mathbb{Q}_{l} \hookrightarrow H_{\text{\'et}}^{2}(\tilde{S}, \mathbb{Q}_{l})(1)
    \end{equation}
    of finite-dimensional vector spaces over $\mathbb{Q}_l$.
\end{thm}

Let $F: S_k \to S_k$ denote the absolute Frobenius, which acts as the identity on the points and by $f \mapsto f^p$ on the structure sheaf.
Set $\varphi:=F^{r}$ and let $\varphi^{(i)}$ denote the automorphism on $H_{\text{\'et}}^{i}(\tilde{S}, \mathbb{Q}_l)$ induced by $\varphi \times 1$ acting on $S_k \times_{\Spec k} \Spec \overline{k} \cong \tilde{S}$.

\begin{cor}{(\cite[Corollary 6.4.]{ref:vanluijk2007})}
    \label{cor:ns_upper_bound}
    The ranks of $\NS (\overline{S})$ and $\NS (\tilde{S})$ are bounded from above by the number of eigenvalues $\lambda$ of $\varphi^{(2)}$ for which $\lambda/q$ is a root of unity, counted with multiplicity.
\end{cor}

\begin{rem}{(\cite[Remark 6.5.]{ref:vanluijk2007})}
    Tate's conjecture states that the upper bound mentioned in Corollary~\ref{cor:ns_upper_bound} is actually equal to the rank of $\NS (\tilde{S})$.
    Tate's conjecture has been proven for elliptic K3 surfaces.
\end{rem}

Now we want to calculate the characteristic polynomial $\chara (\varphi^{(2)})$.
Beforehand, we recall the Lefschetz trace formula.

\begin{thm}
    \begin{equation}
        \# \tilde{S}(\mathbb{F}_{q^{m}}) = \sum_{i = 0}^{n} ( - 1)^{i} \Tr((\varphi^{(i)})^{m})
    \end{equation}
\end{thm}

\begin{cor}
    \label{cor:lefschetz}
    \begin{equation}
        \Tr ((\varphi^{(2)})^{m}) = \# \tilde{S}(\mathbb{F}_{q^{m}}) - 1 - q^{2m}
    \end{equation}
\end{cor}
\begin{proof}
    \begin{equation}
        \dim H_{\text{\'et}}^{1}(\tilde{S}, \mathbb{Q}_l) = \dim H_{\text{\'et}}^{3}(\tilde{S}, \mathbb{Q}_l) = 0
    \end{equation}
    and $\varphi^{(4)}$ acts on $H_{\text{\'et}}^{4}(\tilde{S}, \mathbb{Q}_l) \cong \mathbb{Q}_l$ by multiplication by $q^{2}$.
\end{proof}

Let $V$ be the linear subspace of $H_{\text{\'et}}^{2}(\tilde{S}, \mathbb{Q}_{l})$ generated by the components of the singular fibers and by the zero section and $W = H_{\text{\'et}}^{2}(\tilde{S}, \mathbb{Q}_l) / V$, then
\begin{equation}
    \dim V = \sum_{v \in R} (m_{v} - 1) + 2.
\end{equation}
By the multiplicativity of the characteristic polynomial, we have
\begin{equation}
    \chara (\varphi^{(2)}) = \chara (\varphi^{(2)} | V) \cdot \chara (\varphi^{(2)}_{W})
\end{equation}
and
\begin{equation}
    \Tr ((\varphi^{(2)})^{m}) = \Tr ((\varphi^{(2)} | V)^{m}) + \Tr ((\varphi^{(2)}_{W})^{m}) \label{eq:tr-composition}
\end{equation}
for any $m \in \mathbb{Z}$, where $\varphi^{(2)}_W: W \to W$ is induced by $\varphi^{(2)}$.
Since $\varphi^{(2)}$ acts on $V$ by multiplication by $q$, we have
\begin{equation}
    \chara (\varphi^{(2)} | V) = (x - q)^{\dim V}.
\end{equation}

As for the characteristic polynomial of $\varphi^{(2)}_{W}$, let $t_{m} := \Tr((\varphi^{(2)}_{W})^{m})$, then $\chara (\varphi^{(2)}_{W})$ is the polynomial part of
\begin{equation}
    \frac{x^{\dim W}}{\exp \left( \sum_{m = 1}^{\infty} \frac{t_{m}}{m} x^{-m} \right)} = x^{\dim W}\left( 1 + t_{1} x^{-1} + \frac{t_{1}^{2} - t^{2}}{2} x^{-2} + \frac{-t_{1}^{3} + 3 t_{1} t_{2} - 2 t_{3}}{6} x^{-3} + \cdots\right).
\end{equation}
Here, by \eqref{eq:tr-composition} and Corollary~\ref{cor:lefschetz}, we have
\begin{equation}
    t_{m} = \# \tilde{S}(\mathbb{F}_{q^{m}}) - 1 - q^{2m} - \dim V \cdot q^{m}.
\end{equation}

\begin{lem}{(\cite[Theorem 4, Part III]{ref:mumford2004})}
    \label{lem:k3-betti}
    If $\tilde{S}$ is a K3 surface, then the second Betti number of $\tilde{S}$ is $22$.
\end{lem}

\begin{thm}
    \begin{equation}
        \rank E_{0,u}^{(1 + 3u)}(\overline{\mathbb{Q}}(u)) = 0
    \end{equation}
\end{thm}
\begin{proof}
We denote by $S=\mathcal{E}_{0,u}^{(1 + 3u)} \to \mathbb{P}^1$ the elliptic surface with the generic fiber $E_{0,u}^{(1 + 3u)}$.

Table~\ref{tab:E_{0,u}^{(1 + 3u)}} より K3 なので Lemma~\ref{lem:k3-betti} より
\begin{equation}
    \dim_{\mathbb{Q}_{l}} H_{\text{\'et}}^{2}(\tilde{S}, \mathbb{Q}_{l}) = 22
\end{equation}
である.
$V$ is of rank 19, on which the Frobenius automorphism acts by multiplication by $p$.

\begin{equation}
    \chara (\varphi^{(2)} | V) = (x - 5)^{19}
\end{equation}

Note that all the multiplicative fibers are split in $\mathbb{F}_{5^{m}}$ for $m=1,2,3$.

\begin{equation}
    t_{m} = \# \tilde{S}(\mathbb{F}_{5^{m}}) - 1 - 5^{2m} - 19 \cdot 5^{m}
\end{equation}

\begin{table}[h]
    \centering
    % \caption{Sample Table}
    \begin{tabular}{|c|c|c|c|}
        \hline
        m                                & 1   & 2    & 3     \\
        \hline
        $\# \tilde{S}(\mathbb{F}_{5^m})$ & 120 & 1080 & 18264 \\
        \hline
        $t_m$                            & -1  & -21  & 263   \\
        \hline
    \end{tabular}
    \label{tab:sample}
\end{table}

\begin{equation}
    \chara(\varphi_{W}^{(2)}) = x^{3} + x^{2} + 11 x - 77
\end{equation}

If $\chara (\varphi_{W}^{(2)})$ has a root of the form $x=5\zeta$ for some root of unity $\zeta$, then $\zeta$ is a root of the polynomial
\begin{equation}
    125x^{3} + 25x^{2} + 55 x - 77,
\end{equation}
which is irreducible over $\mathbb{Q}$.
It contradicts the fact that $\zeta$ is an algebraic integer.
By Corollary~\ref{cor:ns_upper_bound}, $\rho(\mathcal{E}_{0,u}^{(1 + 3u)}) \leq 19$.
Then by Theorem~\ref{thm:shioda}, we have
\begin{equation}
    \rank E_{0,u}^{(1 + 3u)}(\overline{\mathbb{Q}}(u)) = 0.
\end{equation}

\end{proof}

\end{document}
