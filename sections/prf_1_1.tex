%! TEX root = ../main.tex
\documentclass[main]{subfiles}

\begin{document}
\chapter{Proof of Theorem~\ref{thm:E_{1,s}}}

\begin{proof}{of Theorem~\ref{thm:E_{1,s}}}
    Let $\mathcal{E}_{1,s} \to \mathbb{P}^{1}$ be the elliptic surface with the generic fiber $E_{1,s}$.
    The discriminant of $E_{1,s}$ is
    \begin{equation}
        \Delta_{E_{1,s}} = 256s^{4} (s + 1)^{4} (s - 1)^{4} (s^{2} + 1)^{4},
    \end{equation}
    and by Tate's algorithm, we have the following table.
    \begin{table}[h]
        \centering
        \caption{Singular fibers of $E_{1,s}$}
        \begin{tabular}{|c|c|c|}
            \hline
            Place             & Type  & $m_v$ \\
            \hline
            $s=0$             & $I_4$ & 4     \\
            $s=\pm 1$         & $I_4$ & 4     \\
            $s=\pm \sqrt{-1}$ & $I_4$ & 4     \\
            $s=\infty$        & $I_4$ & 4     \\
            \hline
        \end{tabular}
    \end{table}

    Then $e(\mathcal{E}_{1,s}) = 24$ and by Theorem~\ref{thm:rho}, we have $\rho(\mathcal{E}_{1,s}) \leq 20$.
    By Shioda-Tate formula (Theorem~\ref{thm:shioda}), we have
    \begin{equation}
        \rank(E_{1,s}) = 0
    \end{equation}

    As for the torsion subgroup, we have
    \begin{equation}
        E_{1,s}(\overline{\mathbb{Q}}(s))[2] = \{\mathcal{O}, (0,0), (4s^{2},0),( - (s^{2} - 1)^{2},0)\},
    \end{equation}
    and we can check by calculation that
    \begin{align}
        2T_1 & = (4s^2,0), \\
        2T_2 & = (0,0).
    \end{align}
    By Theorem~\ref{thm:torsion}, we have $E_{1,s}(\overline{\mathbb{Q}}(s))_ \text{tors} \hookrightarrow (\mathbb{Z} / 4 \mathbb{Z})^{6}$.
    Therefore, we have
    \begin{equation}
        E_{1,s}(\overline{\mathbb{Q}}(s))_ \text{tors} \cong \mathbb{Z} / 4 \mathbb{Z} \oplus \mathbb{Z} / 4 \mathbb{Z}.
    \end{equation}
\end{proof}
\end{document}
