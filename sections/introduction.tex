%! TEX root = ../main.tex
\documentclass[main]{subfiles}

\begin{document}
\chapter{Introduction}
\section{セクション}

\begin{thm}{(Shioda-Tate formula, \cite{ref:naskrecki2013} Theorem 3.4)}
    \label{thm:shioda}
    \begin{equation}
        \rho (S) = 2 + \sum_{v \in R} (m_{v} - 1) + \rank(E(K))
    \end{equation}
    である.
\end{thm}

\begin{thm}
    \begin{equation}
        \label{eq:E_{1s}}
        E_{1,s}: y^{2} = x(x - 4s^{2})(x + (s^{2} - 1)^{2})
    \end{equation}
    の$\overline{\mathbb{Q}}(s)$上のランクは0である.
\end{thm}
\begin{proof}
    \begin{equation}
        \Delta_{E_{1,s}} = 256s^{4} (s + 1)^{4} (s - 1)^{4} (s^{2} + 1)^{4}
    \end{equation}
    
    \begin{table}[h]
        \centering
        \caption{Singular fibers of $E_{1,s}$}
        \begin{tabular}{|c|c|c|}
            \hline
            Place      & Type  & $m_v$ \\
            \hline
            $s=0$      & $I_4$ & 4     \\
            $s=\pm 1$  & $I_4$ & 4     \\
            $s=\pm i$  & $I_4$ & 4     \\
            $s=\infty$ & $I_4$ & 4     \\
            \hline
        \end{tabular}
    \end{table}
    
    \begin{equation}
        e(\mathcal{E}_{1,s}) = 24
    \end{equation}
    したがって $\mathcal{E}_{1,s}$ はK3曲面であり. $\rho(\mathcal{E}_{1,s}) \leq 24$ である.
    Theorem~\ref{thm:shioda} より
    \begin{equation}
        \rank(E_{1,s}) = 0
    \end{equation}
\end{proof}

\begin{thm}
    \begin{equation}
        \label{eq:E_{4t}}
        E_{4,t}: y^{2} = x(x - 4s^{2})(x + (s^{2} - 1)^{2}), s = \frac{2t}{t^{2} - 3}
    \end{equation}
    は
    \begin{equation}
        \left(s^{2} - 1, \sqrt{-1} s(s^{2} - 1) \frac{t^{2} + 3}{t^{2} - 3} \right)
    \end{equation}
    を通る.
    \begin{equation}
        1 \leq \rank E_{4,t}(\overline{\mathbb{Q}}(t)) \leq 2
    \end{equation}
\end{thm}
\begin{proof}
    \begin{equation}
        \Delta_{E_{4,t}} = 4096t^{4}(t - 1)^{4}(t + 1)^{4}(t - 3)^{4}(t + 3)^{4}(t^{2} - 3)^{4}(t^{4} - 2t^{2} + 9)^{4}
    \end{equation}
    \begin{table}[h]
        \centering
        \caption{Singular fibers of $E_{4,t}$}
        \begin{tabular}{|c|c|c|}
            \hline
            Place            & Type  & $m_v$ \\
            \hline
            $t=0$            & $I_4$ & 4     \\
            $t=\pm 1$        & $I_4$ & 4     \\
            $t=\pm 3$        & $I_4$ & 4     \\
            $t=\pm \sqrt{3}$ & $I_4$ & 4     \\
            $t^4-2t^2+9=0$   & $I_4$ & 4     \\
            $t=\infty$       & $I_4$ & 4     \\
            \hline
        \end{tabular}
    \end{table}
    \begin{equation}
        e(\mathcal{E}_{4,t}) = 48
    \end{equation}
    TODO: $\rho(\mathcal{E}_{4,t}) \leq 40$ である.
    Theorem~\ref{thm:shioda} より
    \begin{equation}
        \rank E_{4,t}(\overline{\mathbb{Q}}(t)) \leq 2
    \end{equation}
\end{proof}

上の評価は不十分.生成元は1つしか見つかっていないので,ランクの上界が1であることを示したい.

\begin{thm}{(\cite{ref:naskrecki2013} Proposition 4.1.)}
    \begin{equation}
        \rank E(k(C)) + \rank E^{(u)}(k(C)) = \rank E(k(C)(\sqrt{u})) 
    \end{equation}
\end{thm}

\begin{thm}
    \begin{align}
        E_{4,t}(\overline{\mathbb{Q}}(t)) = E_{1,s}(\overline{\mathbb{Q}}(s)(\sqrt{1 + 3s^{2}})) \\
        E_{1,s}^{(1 + 3s^{2})}: (1 + 3s^{2}) y^{2} = x(x - 4s^{2})(x + (s^{2} - 1)^{2})          \\
        \rank E_{1,s}(\overline{\mathbb{Q}}(s)) + \rank E_{1,s}^{(1 + 3s^{2})}(\overline{\mathbb{Q}}(s)) = \rank E_{4,t}(\overline{\mathbb{Q}}(t))
    \end{align}
\end{thm}
\begin{proof}
    \begin{equation}
        s = \frac{2t}{t^{2} - 3}
    \end{equation}
    を $t$ について解くと
    \begin{equation}
        t = \frac{1 \pm \sqrt{1 + 3s^{2}}}{s}
    \end{equation}
    したがって
    \begin{equation}
        E_{4,t}(\overline{\mathbb{Q}}(t)) = E_{1,s}(\overline{\mathbb{Q}}(s)(\sqrt{1 + 3s^{2}}))
    \end{equation}
\end{proof}

\begin{thm}
    TODO
    \begin{equation}
        \rank E_{1,s}^{(1 + 3s^{2})}(\overline{\mathbb{Q}}(s)) = ?
    \end{equation}
\end{thm}

\begin{proof}
    \begin{equation}
        \Delta(E_{1,s}^{(1 + 3s^{2})}) = (1 + 3s^{2})^{6} \Delta(E_{1,s})
    \end{equation}
    \begin{table}[h]
        \centering
        \caption{Singular fibers of $E_{1,s}^{(1 + 3s^{2})}$}
        \begin{tabular}{|c|c|c|}
            \hline
            Place                        & Type    & $m_v$ \\
            \hline
            $s=0$                        & $I_4$   & 4     \\
            $s=\pm 1$                    & $I_4$   & 4     \\
            $s=\pm i$                    & $I_4$   & 4     \\
            $s=\pm \frac{1}{\sqrt{-3}} $ & $I_0^*$ & 5     \\
            $s=\infty$                   & $I_4$   & 4     \\
            \hline
        \end{tabular}
    \end{table}
    \begin{equation}
        e(\mathcal{E}_{1,s}^{(1 + 3s^{2})}) = 36
    \end{equation}
    TODO
    \begin{equation}
        \rank E_{1,s}^{(1 + 3s^{2})}(\overline{\mathbb{Q}}(s)) = ? (1 or 2)
    \end{equation}
\end{proof}

\end{document}