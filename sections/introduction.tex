%! TEX root = ../main.tex
\documentclass[main]{subfiles}

\begin{document}
\chapter{Introduction}
\section{セクション}

\begin{thm}{(Shioda-Tate formula)}
    \label{thm:shioda}
    \begin{equation}
        \rho (S) = 2 + \sum_{v \in R} (m_{v} - 1) + \rank(E(K))
    \end{equation}
    である.
\end{thm}


\begin{equation}
    \label{eq:E_{1s}}
    E_{1,s}: y^{2} = x(x - 4 * s^{2})(x + (s^{2} - 1)^{2})
\end{equation}

\begin{equation}
    \Delta_{E_{1,s}} = 256s^{4} (s + 1)^{4} (s - 1)^{4} (s^{2} + 1)^{4}
\end{equation}

\begin{table}[h]
    \centering
    \caption{Sample Table}
    \begin{tabular}{|c|c|c|}
        \hline
        s          & type  & $m_v$ \\
        \hline
        $s=0$      & $I_4$ & 4     \\
        \hline
        $s=\pm 1$  & $I_4$ & 4     \\
        \hline
        $s=\pm i$  & $I_4$ & 4     \\
        $s=\infty$ & $I_4$ & 4     \\
    \end{tabular}
    \label{tab:sample}
\end{table}

\begin{equation}
    e(\mathcal{E}_{1,s}) = 24
\end{equation}
したがって $\mathcal{E}_{1,s}$ はK3曲面であり. $\rho(\mathcal{E}_{1,s}) \leq 24$ である.
\ref{thm:shioda} より
\begin{equation}
    \rank(E_{1,s}) = 0
\end{equation}

\begin{equation}
    \label{eq:E_{4t}}
    E_{4,t}: y^{2} = x(x - 4 * s^{2})(x + (s^{2} - 1)^{2}), s = \frac{2t}{t^{2} - 3}
\end{equation}
は
\begin{equation}
    \left(s^{2} - 1, \sqrt{-1} s(s^{2} - 1) \frac{t^{2} + 3}{t^{2} - 3} \right)
\end{equation}
を通る.

\end{document}