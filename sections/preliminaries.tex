%! TEX root = ../main.tex
\documentclass[main]{subfiles}

\begin{document}
\chapter{Preliminaries}

In order to get the lower bound of the rank of the Mordell-Weil group, we can just find points of infinite order.
It is more difficult to get the upper bound of the rank.
The following theorem behaves a key role in the proof of the main theorem.
\begin{thm}{(Shioda-Tate formula, \cite{ref:naskrecki2013} Theorem 3.4)}
    \label{thm:shioda}
    Let $C$ be a smooth irreducible projective curve over an algebraically closed field $k$ and $E$ an elliptic curve over a function field $k(C)$.
    Let $\mathcal{E} \to C$ be the N \'eron model of $E$.
    Let $R \subset C$ be the set of points where the special fiber of $\mathcal{E}$ is singular.
    For each $v \in R$, let $m_{v}$ be the number of components of the special fiber of $\mathcal{E}$ at $v$.
    Let $\rho(\mathcal{E})$ denote the rank of the N \'eron-Severi group of $\mathcal{E}$.
    Then, we have
    \begin{equation}
        \rho (\mathcal{E}) = 2 + \sum_{v \in R} (m_{v} - 1) + \rank(E(k(C))).
    \end{equation}
\end{thm}

We can calculate $R$ and $m_{v}$ by Tate's algorithm.
一方$\rho$については

\begin{equation}
    12 \chi = \sum e(F_{v})
\end{equation}

\begin{thm}
    \label{thm:rho}
    \begin{equation}
        \rho(\mathcal{E}) \leq 10 \chi + 2g
    \end{equation}
\end{thm}

\begin{dfn}
    Let $C$ be a smooth curve over an algebraically closed field $k$.
    Let $E$ be an elliptic curve over a function field $k(C)$ given by the Weierstrass equation
    \begin{equation}
        E: y^{2} = x^{3} + a_{2} x + a_{4} x + a_{6}
    \end{equation}
    where $a_{2}, a_{4}, a_{6} \in k(C)$.
    For a fixed $u \in k(C)^*$, we denote
    \begin{equation}
        E^{(u)}: u y^{2} = x^{3} + a_{2} x + a_{4} x + a_{6}
    \end{equation}
    to be the quadratic twist of $E$ by $u$.
\end{dfn}

In order to make it easier to calculate the rank, we can use the following theorem.

\begin{thm}{(\cite{ref:naskrecki2013} Proposition 4.1.)}
    \begin{equation}
        \rank E(k(C)) + \rank E^{(u)}(k(C)) = \rank E(k(C)(\sqrt{u}))
    \end{equation}
\end{thm}

However, Theorem~\ref{thm:rho} is still not enough to get the upper bound of the rank in our case.

TODO: \'etale cohomology を使う

\end{document}
