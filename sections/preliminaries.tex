%! TEX root = ../main.tex
\documentclass[main]{subfiles}

\begin{document}
\chapter{Preliminaries}

In order to get the lower bound of the rank of the Mordell-Weil group, we can just find points of infinite order.
It is more difficult to get the upper bound of the rank.
The following theorem behaves a key role in the proof of the main theorem.
\begin{thm}{(Shioda-Tate formula, \cite{ref:naskrecki2013} Theorem 3.4)}
    \label{thm:shioda}
    \begin{equation}
        \rho (S) = 2 + \sum_{v \in R} (m_{v} - 1) + \rank(E(K))
    \end{equation}
    である.
\end{thm}

\begin{thm}
    \label{thm:rho}
    rational や K3 のときの $\rho$ はわかっている.
\end{thm}

In order to make it easier to calculate the rank, we can use the following theorem.

\begin{thm}{(\cite{ref:naskrecki2013} Proposition 4.1.)}
    \begin{equation}
        \rank E(k(C)) + \rank E^{(u)}(k(C)) = \rank E(k(C)(\sqrt{u}))
    \end{equation}
\end{thm}

However, Theorem~\ref{thm:rho} is still not enough to get the upper bound of the rank in our case.

TODO: etale cohomology を使う

\end{document}
