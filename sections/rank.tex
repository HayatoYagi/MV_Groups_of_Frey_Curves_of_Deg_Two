%! TEX root = ../main.tex
\documentclass[main]{subfiles}

\begin{document}
\chapter{Ranks}

\begin{proof}{of Theorem~\ref{thm:E_{1,s}}}
    \begin{equation}
        \Delta_{E_{1,s}} = 256s^{4} (s + 1)^{4} (s - 1)^{4} (s^{2} + 1)^{4}
    \end{equation}

    \begin{equation}
        e(\mathcal{E}_{1,s}) = 24
    \end{equation}
    したがって $\mathcal{E}_{1,s}$ はK3曲面であり. $\rho(\mathcal{E}_{1,s}) \leq 20$ である.
    Theorem~\ref{thm:shioda} より
    \begin{equation}
        \rank(E_{1,s}) = 0
    \end{equation}
\end{proof}

\begin{proof}{of Theorem~\ref{thm:E_{2,t}}}
    \begin{equation}
        \Delta_{E_{2,t}} = 4096t^{4}(t - 1)^{4}(t + 1)^{4}(t - 3)^{4}(t + 3)^{4}(t^{2} - 3)^{4}(t^{4} - 2t^{2} + 9)^{4}
    \end{equation}

    \begin{equation}
        e(\mathcal{E}_{4,t}) = 48
    \end{equation}
    TODO: $\rho(\mathcal{E}_{4,t}) \leq 40$ である.
    Theorem~\ref{thm:shioda} より
    \begin{equation}
        \rank E_{2,t}(\overline{\mathbb{Q}}(t)) \leq 2
    \end{equation}
\end{proof}

上の評価は不十分.生成元は1つしか見つかっていないので,ランクの上界が1であることを示したい.
\begin{thm}
    \begin{align}
        E_{2,t}(\overline{\mathbb{Q}}(t)) = E_{1,s}(\overline{\mathbb{Q}}(s)(\sqrt{1 + 3s^{2}})) \\
        E_{1,s}^{(1 + 3s^{2})}: (1 + 3s^{2}) y^{2} = x(x - 4s^{2})(x + (s^{2} - 1)^{2})          \\
        \rank E_{1,s}(\overline{\mathbb{Q}}(s)) + \rank E_{1,s}^{(1 + 3s^{2})}(\overline{\mathbb{Q}}(s)) = \rank E_{2,t}(\overline{\mathbb{Q}}(t))
    \end{align}
    さらに
    \begin{align}
        E_{0,s}: y^{2} = x(x - 4s)(x + (s - 1)^{2})                         \\
        E_{0,s}^{(1 + 3s)}: (1 + 3s) y^{2} = x(x - 4s)(x + (s - 1)^{2})     \\
        E_{0,s}^{(s(1 + 3s))}: s(1 + 3s) y^{2} = x(x - 4s)(x + (s - 1)^{2}) \\
        \rank E_{0,s}^{(1 + 3s)}(\overline{\mathbb{Q}}(s)) + \rank E_{0,s}^{(s(1 + 3s))}(\overline{\mathbb{Q}}(s)) = \rank E_{1,s}^{(1 + 3s^{2})}(\overline{\mathbb{Q}}(s))
    \end{align}
\end{thm}
\begin{proof}
    \begin{equation}
        s = \frac{2t}{t^{2} - 3}
    \end{equation}
    を $t$ について解くと
    \begin{equation}
        t = \frac{1 \pm \sqrt{1 + 3s^{2}}}{s}
    \end{equation}
    したがって
    \begin{equation}
        E_{2,t}(\overline{\mathbb{Q}}(t)) = E_{1,s}(\overline{\mathbb{Q}}(s)(\sqrt{1 + 3s^{2}}))
    \end{equation}
\end{proof}

\begin{thm}
    TODO
    \begin{equation}
        \rank E_{1,s}^{(1 + 3s^{2})}(\overline{\mathbb{Q}}(s)) = ?
    \end{equation}
\end{thm}

\begin{proof}
    \begin{equation}
        \Delta(E_{1,s}^{(1 + 3s^{2})}) = (1 + 3s^{2})^{6} \Delta(E_{1,s})
    \end{equation}
    \begin{equation}
        e(\mathcal{E}_{1,s}^{(1 + 3s^{2})}) = 36
    \end{equation}
    Theorem~\ref{thm:shioda}からは
    \begin{equation}
        \rank E_{1,s}^{(1 + 3s^{2})}(\overline{\mathbb{Q}}(s)) \leq 2
    \end{equation}
    しか分からない.
    K3ですらないので, $H^2$の次元が分からず, reduction を取る方法でも計算が進められない.
    \begin{equation}
        \rank E_{1,s}^{(1 + 3s^{2})}(\overline{\mathbb{Q}}(s)) = ? (1 or 2)
    \end{equation}
\end{proof}


\begin{thm}
    TODO
    \begin{equation}
        \rank E_{0,s}^{(1 + 3s)}(\overline{\mathbb{Q}}(s)) \leq 1
    \end{equation}
\end{thm}
\begin{proof}
    \begin{equation}
        \Delta(E_{0,s}^{(1 + 3s)}) = 256s^{2}(s - 1)^{4}(s + 1)^{4}(3s + 1)^{6}
    \end{equation}
    \begin{equation}
        e(\mathcal{E}_{0,s}^{(1 + 3s)}) = 24
    \end{equation}
    Theorem~\ref{thm:shioda}からは
    \begin{equation}
        \rank E_{0,s}^{(1 + 3s)}(\overline{\mathbb{Q}}(s)) \leq 1
    \end{equation}
\end{proof}

\begin{thm}
    \begin{equation}
        \rank E_{0,s}^{(s(1 + 3s))}(\overline{\mathbb{Q}}(s)) = 1
    \end{equation}
\end{thm}
\begin{proof}
    \begin{equation}
        (s - 1, i(s - 1)) \in E_{0,s}^{(s(1 + 3s))}(\overline{\mathbb{Q}}(s))
    \end{equation}
    より rank は正である.
    \begin{equation}
        \Delta(E_{0,s}^{(s(1 + 3s))}) = 256s^{8}(s - 1)^{4}(s + 1)^{4}(3s + 1)^{6}
    \end{equation}

    上と同様に
    \begin{equation}
        \rank E_{0,s}^{(s(1 + 3s))}(\overline{\mathbb{Q}}(s)) \leq 1
    \end{equation}
\end{proof}

\begin{thm}
    TODO
    \begin{equation}
        \rank E_{1,s}^{(1 + 3s^{2})}(\overline{\mathbb{Q}}(s)) = ?
    \end{equation}
\end{thm}

\begin{proof}
    \begin{equation}
        \Delta(E_{1,s}^{(1 + 3s^{2})}) = (1 + 3s^{2})^{6} \Delta(E_{1,s})
    \end{equation}
    \begin{equation}
        e(\mathcal{E}_{1,s}^{(1 + 3s^{2})}) = 36
    \end{equation}
    Theorem~\ref{thm:shioda}からは
    \begin{equation}
        \rank E_{1,s}^{(1 + 3s^{2})}(\overline{\mathbb{Q}}(s)) \leq 2
    \end{equation}
    しか分からない.
    K3ですらないので, $H^2$の次元が分からず, reduction を取る方法でも計算が進められない.
    \begin{equation}
        \rank E_{1,s}^{(1 + 3s^{2})}(\overline{\mathbb{Q}}(s)) = ? (1 or 2)
    \end{equation}
\end{proof}


\begin{thm}
    TODO
    \begin{equation}
        \rank E_{0,s}^{(1 + 3s)}(\overline{\mathbb{Q}}(s)) \leq 1
    \end{equation}
\end{thm}
\begin{proof}
    \begin{equation}
        \Delta(E_{0,s}^{(1 + 3s)}) = 256s^{2}(s - 1)^{4}(s + 1)^{4}(3s + 1)^{6}
    \end{equation}
    \begin{equation}
        e(\mathcal{E}_{0,s}^{(1 + 3s)}) = 24
    \end{equation}
    Theorem~\ref{thm:shioda}からは
    \begin{equation}
        \rank E_{0,s}^{(1 + 3s)}(\overline{\mathbb{Q}}(s)) \leq 1
    \end{equation}
\end{proof}

\begin{thm}
    \begin{equation}
        \rank E_{0,s}^{(s(1 + 3s))}(\overline{\mathbb{Q}}(s)) = 1
    \end{equation}
\end{thm}
\begin{proof}
    \begin{equation}
        (s - 1, i(s - 1)) \in E_{0,s}^{(s(1 + 3s))}(\overline{\mathbb{Q}}(s))
    \end{equation}
    より rank は正である.
    \begin{equation}
        \Delta(E_{0,s}^{(s(1 + 3s))}) = 256s^{8}(s - 1)^{4}(s + 1)^{4}(3s + 1)^{6}
    \end{equation}

    上と同様に
    \begin{equation}
        \rank E_{0,s}^{(s(1 + 3s))}(\overline{\mathbb{Q}}(s)) \leq 1
    \end{equation}
\end{proof}
\end{document}
