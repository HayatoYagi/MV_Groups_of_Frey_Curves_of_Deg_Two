%! TEX root = ../main.tex
\documentclass[main]{subfiles}

\begin{document}
\chapter{The Generic Rank of $E_{2,t}$}

In order to prove Theorem~\ref{thm:E_{2,t}}, Theorem~\ref{thm:rho} is not enough to get the exact upper bound of the ranks of the \Neron-Severi group.
Actually, we get $\rank E_{2,t}(\overline{\mathbb{Q}}(t)) \leq 2$ from Theorem~\ref{thm:shioda} and Theorem~\ref{thm:rho}.
\begin{table}[h]
    \centering
    \caption{Singular fibers of $E_{2,t}$}
    \begin{tabular}{|c|c|c|}
        \hline
        Place            & Type  & $m_v$ \\
        \hline
        $t=0$            & $I_4$ & 4     \\
        $t=\pm 1$        & $I_4$ & 4     \\
        $t=\pm 3$        & $I_4$ & 4     \\
        $t=\pm \sqrt{3}$ & $I_4$ & 4     \\
        $t^4-2t^2+9=0$   & $I_4$ & 4     \\
        $t=\infty$       & $I_4$ & 4     \\
        \hline
    \end{tabular}
\end{table}
\begin{proof}
    \begin{equation}
        \Delta_{E_{2,t}} = 4096t^{4}(t - 1)^{4}(t + 1)^{4}(t - 3)^{4}(t + 3)^{4}(t^{2} - 3)^{4}(t^{4} - 2t^{2} + 9)^{4}
    \end{equation}

    \begin{equation}
        e(\mathcal{E}_{2,t}) = 48
    \end{equation}
    TODO: $\rho(\mathcal{E}_{2,t}) \leq 40$ である.
    Theorem~\ref{thm:shioda} より
    \begin{equation}
        \rank E_{2,t}(\overline{\mathbb{Q}}(t)) \leq 2
    \end{equation}
\end{proof}
On the other hand, we have only one point of infinite order in $E_{2,t}(\overline{\mathbb{Q}}(t))$.
Now, our goal is to show the upper bound of the rank of $E_{2,t}(\overline{\mathbb{Q}}(t))$ is $1$.

We use another method to estimate the upper bound of the rank of \Neron-Severi group, which we will explain in Chapter~\ref{chap:reduction}.
Beforehand, we express the rank of $E_{2,t}(\overline{\mathbb{Q}}(t))$ in terms of ranks of elliptic curves with lower order coefficients in the Weierstrass equations to make the later computation feasible.

\begin{dfn}
    Let $C$ be a smooth curve over an algebraically closed field $k$.
    Let $E$ be an elliptic curve over a function field $k(C)$ given by the Weierstrass equation
    \begin{equation}
        E: y^{2} = x^{3} + a_{2} x^{2} + a_{4} x + a_{6}
    \end{equation}
    where $a_{2}, a_{4}, a_{6} \in k(C)$.
    For a fixed $u \in k(C)^*$, we denote
    \begin{equation}
        E^{(u)}: u y^{2} = x^{3} + a_{2} x^{2} + a_{4} x + a_{6}
    \end{equation}
    to be the quadratic twist of $E$ by $u$.
\end{dfn}

\begin{thm}{(\cite[Exercise 10.16]{ref:aec})}
    \label{thm:twist}
    Let $E$ be an elliptic curve over a function field $k(C)$ and $u \in k(C)^*$.
    Then, the following equation holds
    \begin{equation}
        \rank E(k(C)(\sqrt{u})) = \rank E(k(C)) + \rank E^{(u)}(k(C)).
    \end{equation}
\end{thm}

\begin{thm}
    Let
    \begin{equation}
        E_{0,u}: y^{2} = x(x - 4u)(x + (u - 1)^{2})
    \end{equation}
    be an elliptic curve over $\overline{\mathbb{Q}}(u)$.
    Then, we have
    \begin{align}
        \rank E_{2,t}(\overline{\mathbb{Q}}(t))                & = \rank E_{1,s}(\overline{\mathbb{Q}}(s)) + \rank E_{1,s}^{(1 + 3s^{2})}(\overline{\mathbb{Q}}(s)), \label{eq:twist1}           \\
        \rank E_{1,s}^{(1 + 3s^{2})}(\overline{\mathbb{Q}}(s)) & = \rank E_{0,u}^{(1 + 3u)}(\overline{\mathbb{Q}}(u)) + \rank E_{0,u}^{(u(1 + 3u))}(\overline{\mathbb{Q}}(u)). \label{eq:twist2}
    \end{align}
    Therefore, we have
    \begin{equation}
        \rank E_{2,t}(\overline{\mathbb{Q}}(t)) = \rank E_{1,s}(\overline{\mathbb{Q}}(s)) + \rank E_{0,u}^{(1 + 3u)}(\overline{\mathbb{Q}}(u)) + \rank E_{0,u}^{(u(1 + 3u))}(\overline{\mathbb{Q}}(u)).
    \end{equation}
\end{thm}
\begin{proof}
    Since solving $s = \frac{2t}{t^{2} - 3}$ for $t$ yields $t = \frac{1 \pm \sqrt{1 + 3s^{2}}}{s}$, we have
    \begin{equation}
        E_{2,t}(\overline{\mathbb{Q}}(t)) = E_{1,s}(\overline{\mathbb{Q}}(s)(\sqrt{1 + 3s^{2}}))
    \end{equation}
    By Theorem~\ref{thm:twist}, we get \eqref{eq:twist1}.
    Similarly, $E_{1,s}$ is obtained by substituting $u = s^{2}$ into $E_{0,u}$, so we have
    \begin{equation}
        E_{1,s}^{(1 + 3s^{2})}(\overline{\mathbb{Q}}(s)) = E_{0,u}^{(1 + 3u)}(\overline{\mathbb{Q}}(u)(\sqrt{u})),
    \end{equation}
    then we get \eqref{eq:twist2}.
\end{proof}

% \begin{thm}
    % TODO
    % \begin{equation}
        % \rank E_{1,s}^{(1 + 3s^{2})}(\overline{\mathbb{Q}}(s)) = ?
        %\end{equation}
    % \end{thm}

% \begin{proof}
    % \begin{equation}
        % \Delta(E_{1,s}^{(1 + 3s^{2})}) = (1 + 3s^{2})^{6} \Delta(E_{1,s})
        %\end{equation}
    % \begin{table}[h]
        % \centering
        % \caption{Singular fibers of $E_{1,s}^{(1 + 3s^{2})}$}
        % \begin{tabular}{|c|c|c|}
            % \hline
            % Place                        & Type    & $m_v$ \\
            % \hline
            % $s=0$                        & $I_4$   & 4     \\
            % $s=\pm 1$                    & $I_4$   & 4     \\
            % $s=\pm \sqrt{-1}$                    & $I_4$   & 4     \\
            % $s=\pm \frac{1}{\sqrt{-3}} $ & $I_0^*$ & 5     \\
            % $s=\infty$                   & $I_4$   & 4     \\
            % \hline
            %\end{tabular}
        % \end{table}
    % \begin{equation}
        % e(\mathcal{E}_{1,s}^{(1 + 3s^{2})}) = 36
        %\end{equation}
    % Theorem~\ref{thm:shioda}からは
    % \begin{equation}
        % \rank E_{1,s}^{(1 + 3s^{2})}(\overline{\mathbb{Q}}(s)) \leq 2
        %\end{equation}
    % しか分からない.
    % K3ですらないので, $H^2$の次元が分からず, reduction を取る方法でも計算が進められない.
    % \begin{equation}
        % \rank E_{1,s}^{(1 + 3s^{2})}(\overline{\mathbb{Q}}(s)) = ? (1 or 2)
        %\end{equation}
    % \end{proof}


\begin{thm}
    TODO
    \begin{equation}
        \rank E_{0,u}^{(1 + 3u)}(\overline{\mathbb{Q}}(u)) \leq 1
    \end{equation}
\end{thm}
\begin{proof}
    \begin{equation}
        \Delta(E_{0,u}^{(1 + 3u)}) = 256u^{2}(u - 1)^{4}(u + 1)^{4}(3u + 1)^{6}
    \end{equation}
    \begin{table}[h]
        \centering
        \caption{Singular fibers of $E_{0,u}^{(1 + 3u)}$}
        \begin{tabular}{|c|c|c|}
            \hline
            Place            & Type    & $m_v$ \\
            \hline
            $u=0$            & $I_2$   & 2     \\
            $u=\pm 1$        & $I_4$   & 4     \\
            $u=-\frac{1}{3}$ & $I_0^*$ & 5     \\
            $u=\infty$       & $I_2^*$ & 7     \\
            \hline
        \end{tabular}
        \label{tab:E_{0,u}^{(1 + 3u)}}
    \end{table}
    \begin{equation}
        e(\mathcal{E}_{0,u}^{(1 + 3u)}) = 24
    \end{equation}
    Theorem~\ref{thm:shioda}からは
    \begin{equation}
        \rank E_{0,u}^{(1 + 3u)}(\overline{\mathbb{Q}}(u)) \leq 1
    \end{equation}
\end{proof}

\begin{thm}
    \begin{equation}
        \rank E_{0,u}^{(u(1 + 3u))}(\overline{\mathbb{Q}}(u)) = 1
    \end{equation}
\end{thm}
\begin{proof}
    \begin{equation}
        (u - 1, \sqrt{-1}(u - 1)) \in E_{0,u}^{(u(1 + 3u))}(\overline{\mathbb{Q}}(u))
    \end{equation}
    より rank は正である.
    \begin{equation}
        \Delta(E_{0,u}^{(u(1 + 3u))}) = 256u^{8}(u - 1)^{4}(u + 1)^{4}(3u + 1)^{6}
    \end{equation}
    \begin{table}[h]
        \centering
        \caption{Singular fibers of $E_{0,u}^{(u(1 + 3u))}$}
        \begin{tabular}{|c|c|c|}
            \hline
            Place            & Type    & $m_v$ \\
            \hline
            $u=0$            & $I_2^*$ & 7     \\
            $u=\pm 1$        & $I_4$   & 4     \\
            $u=-\frac{1}{3}$ & $I_0^*$ & 5     \\
            $u=\infty$       & $I_2$   & 2     \\
            \hline
        \end{tabular}
    \end{table}
    上と同様に
    \begin{equation}
        \rank E_{0,u}^{(u(1 + 3u))}(\overline{\mathbb{Q}}(u)) \leq 1
    \end{equation}
\end{proof}

\end{document}
