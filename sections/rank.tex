%! TEX root = ../main.tex
\documentclass[main]{subfiles}

\begin{document}
\chapter{Ranks}

\begin{thm}
    TODO
    \begin{equation}
        \rank E_{1,s}^{(1 + 3s^{2})}(\overline{\mathbb{Q}}(s)) = ?
    \end{equation}
\end{thm}

\begin{proof}
    \begin{equation}
        \Delta(E_{1,s}^{(1 + 3s^{2})}) = (1 + 3s^{2})^{6} \Delta(E_{1,s})
    \end{equation}
    \begin{equation}
        e(\mathcal{E}_{1,s}^{(1 + 3s^{2})}) = 36
    \end{equation}
    Theorem~\ref{thm:shioda}からは
    \begin{equation}
        \rank E_{1,s}^{(1 + 3s^{2})}(\overline{\mathbb{Q}}(s)) \leq 2
    \end{equation}
    しか分からない.
    K3ですらないので, $H^2$の次元が分からず, reduction を取る方法でも計算が進められない.
    \begin{equation}
        \rank E_{1,s}^{(1 + 3s^{2})}(\overline{\mathbb{Q}}(s)) = ? (1 or 2)
    \end{equation}
\end{proof}


\begin{thm}
    TODO
    \begin{equation}
        \rank E_{0,s}^{(1 + 3s)}(\overline{\mathbb{Q}}(s)) \leq 1
    \end{equation}
\end{thm}
\begin{proof}
    \begin{equation}
        \Delta(E_{0,s}^{(1 + 3s)}) = 256s^{2}(s - 1)^{4}(s + 1)^{4}(3s + 1)^{6}
    \end{equation}
    \begin{equation}
        e(\mathcal{E}_{0,s}^{(1 + 3s)}) = 24
    \end{equation}
    Theorem~\ref{thm:shioda}からは
    \begin{equation}
        \rank E_{0,s}^{(1 + 3s)}(\overline{\mathbb{Q}}(s)) \leq 1
    \end{equation}
\end{proof}

\begin{thm}
    \begin{equation}
        \rank E_{0,s}^{(s(1 + 3s))}(\overline{\mathbb{Q}}(s)) = 1
    \end{equation}
\end{thm}
\begin{proof}
    \begin{equation}
        (s - 1, \sqrt{-1}(s - 1)) \in E_{0,s}^{(s(1 + 3s))}(\overline{\mathbb{Q}}(s))
    \end{equation}
    より rank は正である.
    \begin{equation}
        \Delta(E_{0,s}^{(s(1 + 3s))}) = 256s^{8}(s - 1)^{4}(s + 1)^{4}(3s + 1)^{6}
    \end{equation}

    上と同様に
    \begin{equation}
        \rank E_{0,s}^{(s(1 + 3s))}(\overline{\mathbb{Q}}(s)) \leq 1
    \end{equation}
\end{proof}
\end{document}
