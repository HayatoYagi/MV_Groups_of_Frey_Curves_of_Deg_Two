%! TEX root = ../main.tex
\documentclass[main]{subfiles}

\begin{document}

\chapter{Torsions}

\section{セクション}

\begin{thm}
    \begin{equation}
        E_{4,t}(\overline{\mathbb{Q}}(t))_ \text{tors} = \mathbb{Z} / 4 \mathbb{Z} \times \mathbb{Z} / 4 \mathbb{Z}
    \end{equation}
    \begin{align}
        T_1 & = (2s(s+1)^2,2s(s+1)^2(s^2+1))   \\
        T_2 & = (2is(s^2-1),2is(s+i)^2(s^2-1))
    \end{align}
    で生成される.
\end{thm}
\begin{proof}
    \begin{equation}
        E_{4,t}(\overline{\mathbb{Q}}(t))[2] = E_{1,s}(\overline{\mathbb{Q}}(s))[2] = \{\mathcal{O}, (0,0), (4s^{2},0),( - (s^{2} - 1)^{2},0)\}
    \end{equation}
    \begin{align}
        2T_1 & = (4s^2,0) \\
        2T_2 & = (0,0)
    \end{align}
    \cite{ref:naskrecki2013} の Lem.3.5 より
    \begin{equation}
        E_{4,t}(\overline{\mathbb{Q}}(t))_ \text{tors} \hookrightarrow (\mathbb{Z} / 4 \mathbb{Z})^{12}
    \end{equation}
    なので位数8の点は存在しない.
\end{proof}

\begin{rem}
    これは
    \begin{equation}
        E_{1,s}(\mathbb{Q}(s))_ \text{tors} = \mathbb{Z} / 2 \mathbb{Z} \times \mathbb{Z} / 4 \mathbb{Z}
    \end{equation}
    の別証明になっている.
\end{rem}

\end{document}
